\documentclass[t, notes, xcolor=table]{beamer}

\usepackage{wrapfig}
\usepackage{float}
% For tabs in verbatim
\usepackage{fancyvrb}

% Adjust position of the image
\usepackage[export]{adjustbox}

% set fonts
\usefonttheme{professionalfonts} % using non standard fonts for beamer
\usepackage{txfonts,mathptmx}

% set indend spacing for first and second level indentation
\setlength{\leftmargini}{0.5cm}
\setlength{\leftmarginii}{0.5cm}
\setlength{\leftmarginiii}{0.5cm}

% Set circles for bullets 
\setbeamertemplate{itemize items}[circle]

% colors
\usepackage{xcolor}

% multiple columns
\usepackage{multicol}

% todo lists
\usepackage{pifont}
\usepackage{amssymb}

% increase space between text and frame name
\addtobeamertemplate{frametitle}{}{\vspace{0.5em}}

%Information to be included in the title page:
\title{Making Procedural Statements}
\author{Nikola Petrovic}
\institute{University of Belgrade, School of Electrical Engineering}
\date{2022}



\begin{document}

\frame{\titlepage}

%%%%%%%%%%%%%%%%%%%%%%%%%%%%%%%%%%%%%%%%%%%%%%%%%%%%%%%%%%%%
\begin{frame}
\frametitle{Module Objective}

In this module we will describe design behaviour procedurally.
\textbf{Topics:}
\begin{itemize}
\item Procedural blocks review
\item Making procedural statements
\item Making conditional statements
\item Making case statements
\item Making loop statements
\end{itemize}

\end{frame}
\note{
Out objective is to appropriately and effectively procedurally describe design behaviour. To do that, we need to know generally about procedural blocks and procedural statements, and more specifically about branching procedural statements. The Verilog behavioural modeling constructs are very similar to the C programming language. They differ in subtle ways that can sometimes be irksome.
}

%%%%%%%%%%%%%%%%%%%%%%%%%%%%%%%%%%%%%%%%%%%%%%%%%%%%%%%%%%%%
\begin{frame}
\frametitle{Describing Module Behaviour}

\begin{itemize}
\item The \textcolor{purple}{initial} construct starts execution at the start of the simulation, and when complete, terminates the process.
\item The \textcolor{purple}{always} construct starts execution at the start of the simulation, and when complete, executes again.

\begin{figure}
    \includegraphics[width=0.95\textwidth]{img/06_mod_beh.png}
\end{figure}
\end{itemize}
\end{frame}
\note{
\scriptsize{
Event simulation relies upon processes. A process is an object that reacts to input events and generates output events. The continuous assignment that we have already seen is a process - the simulator reacts to its input transitions to calculate new output value and transitions the net to that new value.
\newline

Verilog has other kinds of processes - two most obvious to the user are the \textbf{initial} construct and the \textbf{always} construct, frequently referred to as the initial and the always block. The \textit{initial} and \textit{always} keywords are not statements themselves - they apply to their following statement and make it execute as a process. Their following statement is usually a statement block, e.g., statements between the \textit{begin} and \textit{end} keywords, but does not have to be.
\begin{itemize}
\item The simulator executes each \textit{initial} process at the start of the simulation, and upon completing its execution, terminates the process.
\item The simulator executes each \textit{always} process at the start of the simulation, and upon completing its execution, executes it again.
\end{itemize}

}
}

%%%%%%%%%%%%%%%%%%%%%%%%%%%%%%%%%%%%%%%%%%%%%%%%%%%%%%%%%%%%
\begin{frame}
\frametitle{Synchronizing Module Behaviours}
\scriptsize{
\begin{multicols}{2}
\begin{itemize}
\item Procedures "block" upon encountering an event control
\item Procedures "unblock" upon occurrence of any of the listed events.
\item An event control can reference:
\begin{itemize}
\scriptsize{
	\item A single event identifier
	\item An event expression
	\item The "$*$" wildcard operator
}
\end{itemize}

\end{itemize}
\vfill
\columnbreak
\begin{figure}
    \includegraphics[width=0.45\textwidth]{img/06_sync.png}
\end{figure}
\end{multicols}
}
\end{frame}
\note{
\scriptsize{
If an \textbf{always} construct has no constructs to block it execution, then when the simulator started executing it at the start of simulation, execution would continue in a tight loop forever and the simulator would appear to hang.
\newline

Verilog provides procedural timing controls for "stepping" execution of procedural blocks. The most common of these is the event control. An event control starts with the \textbf{at (@)} character and then follows with either a wildcard character, a single event identifier, or a parenthesized event expression. The event expression can be a list of event expressions separated by the \textbf{or} keyword or by the comma (,) character. Both are shown in the example. The \textbf{or} keyword in an event expression is a separator between event expressions and is not an operator in the usual sense. We can further qualify an expression with the \textbf{posedge} or \textbf{negedge} keywords.
\newline

A timing control is not itself a statement, but precedes a statement, and in the case of an event control, blocks execution of that statement and subsequent sequential statements until on of those events occur.

}
}


%%%%%%%%%%%%%%%%%%%%%%%%%%%%%%%%%%%%%%%%%%%%%%%%%%%%%%%%%%%%
\begin{frame}
\frametitle{Interactions Between Behaviours}
\begin{figure}
    \includegraphics[width=0.95\textwidth]{img/06_interaction.png}
\end{figure}
\end{frame}
\note{
\scriptsize{
A non-trivial design usually contains several modules, and a module at the RT level typically contains procedural blocks:
\begin{itemize}
\item Each sequential procedural block executes its statement sequentially like a conventional programming language.
\item Multiple blocks execute concurrently, like hardware.
\item Multiple blocks communicate with each other using events, nets and variables.
\end{itemize}

The capability to have many procedural blocks communicating concurrently with each other is the basic model which Verilog uses to describe hardware.
}
}

%%%%%%%%%%%%%%%%%%%%%%%%%%%%%%%%%%%%%%%%%%%%%%%%%%%%%%%%%%%%
\begin{frame}
\frametitle{Making Procedural Assignments}
\scriptsize{
\begin{multicols}{2}
\begin{itemize}
\item Procedural assignments are assignments made inside procedures
\item We make procedural assignments only to variables!
\end{itemize}
\vfill
\columnbreak
\begin{figure}
    \includegraphics[width=0.45\textwidth]{img/06_make_assign.png}
\end{figure}
\end{multicols}
}
\end{frame}
\note{
\scriptsize{
The \textbf{initial} construct and the \textbf{always} construct are frequently referred to as \textbf{procedural blocks}, though as we have seen, whey can apply to an atomic statement. Statements within a \textit{procedural block} are procedural statements, and most of our procedural statements will be assignments.
\begin{itemize}
\item Recall that a continuous assignment is to a net and is its own process, so only exists \textbf{outside} any procedural blocks.
\item A procedural assignment is to a variable and is not its own process, so only exists \textbf{inside} a procedural block.
\item The right operand of either assignment can contain constants, variables and nets, as on the right side only their values are used.
\end{itemize}

}
}

%%%%%%%%%%%%%%%%%%%%%%%%%%%%%%%%%%%%%%%%%%%%%%%%%%%%%%%%%%%%
\begin{frame}
\frametitle{Making Conditional Statements}
\scriptsize{
\begin{multicols}{2}
\begin{itemize}
\item \textbf{if} is a conditional statement
\item \textbf{if} tests the condition in sequence
\item Conditions can overlap
\item Executes statement associated with first known true condition
\item Statement can be null ";"
\end{itemize}
\vfill
\columnbreak
\begin{figure}
    \includegraphics[width=0.45\textwidth]{img/06_cond.png}
\end{figure}
\end{multicols}
}
\end{frame}
\note{
\tiny{
\textbf{if} statement syntax:

if (expression)

\ \ statement
  
[\{else if (expression)

\ \ statement\}]
  
[else statement]
\newline

The conditional statement is a two-way branch. I the conditional expression evaluates to a known non-zero value then the first branch executes and otherwise the second branch executes if it exists. The conditional expression can be multiple bits.
\newline

As with other programming languages, in nested conditional statements, every \textbf{else} keyword is associated with the nearest previous \textbf{if} keyword that does not already have an \textit{else} associated with it. If this association displeases us, then we can force association by using \textbf{begin-end} sequential blocks, or we can just toss in the else keyword where needed with a null following statement.
\newline

In this example, execution of the procedural block blocks at the event control until a transition occurs on a least one of the module inputs. When a transition occurs, the conditional statement executes. It evaluates the \textit{sel} signal. if the value is exactly 3'b000, then it executes the first branch, which is an assignment of a to y. If the value is anything else, it executes the second branch, which is another conditional statement. This conditional statement again evaluates the sel signal. if the value is less than or equal to 3'b101, then it executes first branch, otherwise it executes the second branch. Note that such a chain of conditional statements implies priority. If the \textit{sel} signal is exactly 3'b000, then the second test for less than 3'b101 is not made, even though the value would match that second test if it were made.

}
}

%%%%%%%%%%%%%%%%%%%%%%%%%%%%%%%%%%%%%%%%%%%%%%%%%%%%%%%%%%%%
\begin{frame}
\frametitle{Conditional Statement Syntax}
\scriptsize{
\begin{multicols}{2}
\begin{figure}
    \includegraphics[width=0.25\textwidth, left]{img/06_cond_syn0.png}
\end{figure}
\vfill
\columnbreak
\textcolor{purple}{if} branch executes if condition is known true.
\end{multicols}
\begin{multicols}{2}
\begin{figure}
    \includegraphics[width=0.25\textwidth, left]{img/06_cond_syn1.png}
\end{figure}
\vfill
\columnbreak
\textcolor{purple}{if} branch executes if condition is known true. \textcolor{purple}{else} branch executes if condition is false or unknown.
\end{multicols}
\begin{multicols}{2}
\begin{figure}
    \includegraphics[width=0.32\textwidth, left]{img/06_cond_syn2.png}
\end{figure}
\vfill
\columnbreak
\textcolor{purple}{if} branch executes if condition  is known true. \textcolor{purple}{else if} branch executes for alternative condition known true. \textcolor{purple}{else} branch executes if no condition known true.
\end{multicols}
}
\end{frame}
\note{
The 1st illustration omits the second branch.
\newline

The 2nd illustration includes a new branch.
\newline

The 3rd illustration is not a new syntax. Unlike some programming languages, Verilog does not have the \textbf{elsif} keyword, so this is simply chaining two conditional statements.
}

%%%%%%%%%%%%%%%%%%%%%%%%%%%%%%%%%%%%%%%%%%%%%%%%%%%%%%%%%%%%
\begin{frame}
\frametitle{Making Case Statements: case}
\scriptsize{
\begin{multicols}{2}
\begin{itemize}
\item The \textcolor{purple}{case} statement is a multi-way prioritized branching statements.
\item \textcolor{purple}{case} item match expressions can overlap.
\item Compares item match expressions to case expression in the order that they appear.
\begin{itemize}
\scriptsize{
	\item uses case equality comparison
	\begin{itemize}
	\scriptsize{
		\item Bitwise 0,1,Z,X comparison	
	}
	\end{itemize}
	\item Executes statement associated with first match
	\item Executes optional default item if no other item matched
	\begin{itemize}
	\scriptsize{
		\item In this example, inputs containing Z or X	
	}
	\end{itemize}
}
\end{itemize}
\end{itemize}
\vfill
\columnbreak
\begin{figure}
    \includegraphics[width=0.45\textwidth]{img/06_case.png}
\end{figure}
\end{multicols}
}
\end{frame}
\note{
\tiny{
case (expression)

expression \{, expression\} : statement

\{expression \{, expression \} : statement \}

[default [:] statement]

endcase
\newline

The \textbf{case} statement is a multi-way prioritized branching statement. The case statement evaluates an expression and then evaluates a sequence of match expressions and executes the statement associated with only the first matching expression. 
The case statement considers high-impedance (z) and unknown (x) values and matches expressions that have a high-impedance or unknown bit in the same position in both expressions. We can comma separate multiple match expression associated with the same following statement. The following statement is often a statement block but does not have to be. We can optionally provide a \textbf{default} match item to be matched when no other match expression matches. it is common to place the default match item at the case statement end, but we can place it anywhere. We can have at most one default match item in each case statement.
\newline

In this example, execution of the procedural block is blocked at the event control until a transition occurs on at least on of the module inputs. When a transition occurs, the \textit{case} statement executes. It evaluates the "sel" signal. if the value is 0 then it executes the first branch. If the value is between 1 and 5 then it executes the second branch. If the value is between 6 and 7 then it executes the third branch. If the value is anything else, it executes the default branch, which sets the output unknown. As an unknown value does not infer any particular hardware, the hardware has no implementation of this default branch.

}
}


%%%%%%%%%%%%%%%%%%%%%%%%%%%%%%%%%%%%%%%%%%%%%%%%%%%%%%%%%%%%
\begin{frame}
\frametitle{Making Case Statements: casex}
\tiny{
\begin{multicols}{2}
Threats Z,X and ? characters as "don't care" bit positions in either:
\begin{itemize}
\item case expression (sel)
\item case item expression 3'b0XX
\end{itemize}
Lets you group match values for more concise description:
\begin{itemize}
\item Encoders, decoders, etc.
\item Not very useful in testbenches
\item Not a preferred method usually
\end{itemize}

\textcolor{purple}{casex} treats X in case expression as meaningful in uninitialized logic:
\begin{itemize}
\item Hides initialization problems
\item Not recommended!
\end{itemize}
\textcolor{purple}{casex} can't match unknown or high impedance values coming out of the DUT, so it's less useful in testbenches
\vfill
\columnbreak
\begin{figure}
    \includegraphics[width=0.45\textwidth]{img/06_casex.png}
\end{figure}
\end{multicols}
}
\end{frame}
\note{
\tiny{
The \textbf{casex} statement is a variation of the \textit{case} statement that lets you specify the bit positions it should not compare.
\newline

The \textit{casex} statement treats the x,z, and question mark (?) characters as \textit{don't care} bit values. Recall that in a literal constant the ? character is equivalent to the z character and that only the based literal constants can use the x,z, and ? characters.
\newline

The \textit{casex} statement does not compare bit positions of either the case expression or the case match items that have any of these "don't care" values.
\newline

By convention, we might want to use the ? character to indicate \textit{don't care} bit positions in literal constants to avoid confusion with the high-impedance and unknown values. However, the \textit{casex} statement still considers bit positions containing high-impedance (z) or unknown (x) values in expressions of nets and variables to be position it should not match, so a \textit{casex} statement in a testbench cannot test high-impedance or unknown values coming from the DUT.
\newline

This example is functionally identical to the previous example, but uses "don't care" bit positions to represent a range of matching values.

}
}

%%%%%%%%%%%%%%%%%%%%%%%%%%%%%%%%%%%%%%%%%%%%%%%%%%%%%%%%%%%%
\begin{frame}
\frametitle{Making Case Statements: casez}
\tiny{
\begin{multicols}{2}
Treat Z and ? characters as "don't care" bit positions in: 
\begin{itemize}
\item case expression (sel)
\item case item expression 3'b0ZZ
\end{itemize}
Performs definitive match for X
\begin{itemize}
\item useful in testbenches!
\end{itemize}
\textcolor{purple}{casez} is preferable to \textcolor{purple}{casex}
\begin{itemize}
\item X in case expression is not wildcard
\item Safer for uninitialized RTL code
\end{itemize} 
\vfill
\columnbreak
\begin{figure}
    \includegraphics[width=0.45\textwidth]{img/06_casez.png}
\end{figure}
\end{multicols}
}
\end{frame}
\note{
\scriptsize{
The \textbf{casez} statement is a variation of the \textbf{casex} statement and treats only the z and question mark ? as "don't care" bit values and performs a definitive match for unknown (x) values. In a testbench, to test the existence of unknown values from the DUT, we should use the \textit{casez} statement instead of the \textit{casex} statement.
\newline

This example is functionally identical to the previous example, but uses the z character instead of the x character to represent the don't care bit positions.

}
}

%%%%%%%%%%%%%%%%%%%%%%%%%%%%%%%%%%%%%%%%%%%%%%%%%%%%%%%%%%%%
\begin{frame}
\frametitle{Alternative Case Statement Form}
\scriptsize{
\begin{multicols}{2}
\begin{itemize}
\item \textcolor{purple}{case} item match can be an expression as well
\item Case item match expression does not have to be a constant value!
\item Can use case statement to match \textit{variable} expressions
\item Case items are prioritized according to the order
\end{itemize}
\vfill
\columnbreak
\begin{figure}
    \includegraphics[width=0.45\textwidth]{img/06_alt.png}
\end{figure}
\end{multicols}
}
\end{frame}
\note{
\tiny{
Alternate \textbf{case} statement syntax:

case (expression)

expression \{ , expression \} : statement

\{expression \{ , expression \} : statement \}

[ default [:] statement ]

endcase
\newline

Unlike most programming languages, the Verilog case item expression does not have to be a constant.
\newline

This example is functionally identical to the previous example, but tests which match expression first matches the value 1, that is, which on is the first "true" expression.
\newline

case (1'b1)

  en\_a : op = a;
  
  en\_b : op = b;
   
  en\_c : op = c;
    
endcase
\newline

In the above code, the case statement en\_a has a highest priority. The statement checks for en\_a first. If this is 1'b1, then the output op is a. Otherwise it checks for en\_b. If en\_b is 1'b1 then output op is equal to b. When both en\_a and en\_b are not 1'b1, then en\_c is checked. If en\_c is 1'b1 then output op = c.

}
}
%%%%%%%%%%%%%%%%%%%%%%%%%%%%%%%%%%%%%%%%%%%%%%%%%%%%%%%%%%%%
\begin{frame}
\frametitle{Making Loop Statements: while}
\scriptsize{
\begin{multicols}{2}
\begin{itemize}
\item \textcolor{purple}{while} loop iterates and executes its following statement while the expression is known and non-zero
\item Must previously declare any variables used in expression
\item While expression is known and true, executes expression.
\end{itemize}
\vfill
\columnbreak
\begin{figure}
    \includegraphics[width=0.45\textwidth]{img/06_while.png}
\end{figure}
\end{multicols}
}
\end{frame}
\note{
\scriptsize{
\textcolor{purple}{while} (expression)
statement
\newline

The \textbf{while} loop executes its following statement while the expression is known and non-zero. The following statement is usually a statement block but does not have to be. if the expression is not known or is zero, then the following statement never executes.
\newline

The 1st illustration executes a statement block while the count is known and less then 10. The last statement of the block increments the count. Some statement before the loop presumably initializes the count, bit this is not shown.
\newline

The 2nd illustration counts the number of bits in the tempreg that are 1. While the "tempreg" is known and non-zero, if the least significant bit of the \textit{tempreg} is known and non-zero it increments the count. In any case, each loop iteration shifts the \textit{tempreg} one position to the right.

}
}

%%%%%%%%%%%%%%%%%%%%%%%%%%%%%%%%%%%%%%%%%%%%%%%%%%%%%%%%%%%%
\begin{frame}
\frametitle{Making Loop Statements: for}
\scriptsize{
\begin{multicols}{2}
\begin{itemize}
\item \textcolor{purple}{for} loop requires prior declaration of any variables used in expressions
\item Initialization is done for the loop
\item Condition checked
\item while condition is known true, executes statement(s)
\item Update expression is then executed ad end of each iteration
\end{itemize}
\vfill
\columnbreak
\begin{figure}
    \includegraphics[width=0.45\textwidth]{img/06_for.png}
\end{figure}
\end{multicols}
}
\end{frame}
\note{
\scriptsize{
\textcolor{purple}{for} (initializer exp; condition exp; update exp)
statement
\newline


We can more concisely rewrite a \textit{while} loop as a \textit{for} loop that is not as complex. The \textit{for} loop performs the initial variable assignment and then while the test expression is known and non-zero executes the following statement. The following statement is usually a statement block but does not have to be. After each iteration of its following statement, the \textit{for} loop executes the second variable assignment. The two variable assignments are almost always to a variable that participates in the test expression, but they don't have to be.
\newline

This example calculates the parity of an input vector reg. The for loop initializes the index to 0 ad while the index is less than or equal to 3 it exclusive-ORs the partial result with each  indexed bit of the input, each time incrementing the index value by 1.
\newline

If we think for a moment, we can think of an operator that will do this one assignment without a loop. The same for loop can also be implemented using an equivalent while loop. The syntax is:

Initializer exp;

while (condition exp)

statement;

update exp;


}
}

%%%%%%%%%%%%%%%%%%%%%%%%%%%%%%%%%%%%%%%%%%%%%%%%%%%%%%%%%%%%
\begin{frame}
\frametitle{Making Loop Statements: repeat}
\scriptsize{
\begin{multicols}{2}
\begin{itemize}
\item \textcolor{purple}{repeat} loop executes the statements number of times specified in the loop
\item Must previously declare any variables used in expression
\item For expressed count executes statement
\item The \textbf{repeat} loop is equivalent to a \textbf{for} loop with the index variable automatically declared, initialized and incremented.

\end{itemize}
\vfill
\columnbreak
\begin{figure}
    \includegraphics[width=0.45\textwidth]{img/06_repeat.png}
\end{figure}
\end{multicols}
}
\end{frame}
\note{
\scriptsize{
\textcolor{purple}{repeat} (expression)
statement
\newline

We can more concisely rewrite a for loop as a \textit{repeat} loop, where we do not need to use the index, for example, to select a bit of a vector. A \textit{repeat} loop executes its following statement the number of times specified by its parenthesized expression.
\newline

This example multiplies two 4-bit inputs to produce an 8-bit output. Because it uses a shift multiplier implementation, it does not need a loop index, so uses a \textit{repeat} loop instead of a \textit{for} loop. Equivalent logic can be optained using for loop as well. The syntax would be:
\newline

integer h; // hidden index

for (h = 0; h \textless\ expression; h = h + 1)

statement

}
}

%%%%%%%%%%%%%%%%%%%%%%%%%%%%%%%%%%%%%%%%%%%%%%%%%%%%%%%%%%%%
\begin{frame}[fragile]
\frametitle{Making Loop Statements: forever}
\begin{itemize}
\item \textcolor{purple}{forever} loop executes statement forever
\end{itemize}


\begin{Verbatim}[commandchars=\\\{\}, tabsize=2]
\textcolor{purple}{reg} clock;
\textcolor{purple}{initial}
  \textcolor{purple}{forever}
    \textcolor{magenta}{#}\textcolor{purple}{10} clock = !clock;
\end{Verbatim}
\end{frame}
\note{
\textcolor{purple}{forever} statement;
\newline

Verilog also provides a forever loop for when we need to repeat execution of a statement an essentially infinite number of times or until simulation ends. The same logic can be modeled using the while loop as well. The syntax would be:
\newline

while(1)

statement


}

%%%%%%%%%%%%%%%%%%%%%%%%%%%%%%%%%%%%%%%%%%%%%%%%%%%%%%%%%%%%
\begin{frame}
\frametitle{Module Summary}
\scriptsize{
Now we can appropriately and correctly describe design behaviour procedurally.
\newline

In this module, we studied:
\begin{itemize}
\item The \textbf{initial} and \textbf{always} construct introduce procedural blocks.
\item The \textbf{@} event control blocks further procedural block execution until an event occurs
\item The \textbf{=} procedural assignment that is inside a procedural block we make only to variables
\item The \textbf{if-else}, and \textbf{case/casex/casez} procedural branching statements
\item The \textbf{for, while, repeat} and \textbf{forever} procedural looping statements.
\end{itemize}
}
\end{frame}

%%%%%%%%%%%%%%%%%%%%%%%%%%%%%%%%%%%%%%%%%%%%%%%%%%%%%%%%%%%%
\begin{frame}
\frametitle{Module Review}
\begin{enumerate}
\item If more than one case item expression matches the case expression, which associated statement(s) execute?
\item Can we use the \textbf{case, if} and \textbf{for} statements in continuous assignments.
\end{enumerate}
\end{frame}
\note{
\begin{enumerate}
\item If more than one case item expression matches the case expression, which associated statement(s) execute?
\begin{itemize}
	\item The case expression is compared to the case item expressions in the order they appear and the statement associated with only the first match is executed.
\end{itemize}
\item Can we use the \textbf{case, if} and \textbf{for} statements in continuous assignments.
\begin{itemize}
	\item The \textbf{case, if} and \textbf{for} statements are procedural statements that cannot appear in a continuous assignment.
\end{itemize}
\end{enumerate}

}

%%%%%%%%%%%%%%%%%%%%%%%%%%%%%%%%%%%%%%%%%%%%%%%%%%%%%%%%%%%%
\begin{frame}
\frametitle{Module Exercise 1}
Code the following conditional logic as a procedural block:
\begin{figure}
    \includegraphics[width=0.45\textwidth]{img/06_ex1.png}
\end{figure}
\end{frame}

\note{
\begin{figure}
    \includegraphics[width=0.45\textwidth]{img/06_ex1s.png}
\end{figure}
}

%%%%%%%%%%%%%%%%%%%%%%%%%%%%%%%%%%%%%%%%%%%%%%%%%%%%%%%%%%%%
\begin{frame}
\frametitle{Module Exercise 2}
\begin{figure}
    \includegraphics[width=0.65\textwidth]{img/06_ex2.png}
\end{figure}
\end{frame}
\note{
\begin{figure}
    \includegraphics[width=0.85\textwidth]{img/06_ex2s.png}
\end{figure}
}

%%%%%%%%%%%%%%%%%%%%%%%%%%%%%%%%%%%%%%%%%%%%%%%%%%%%%%%%%%%%
\begin{frame}
\frametitle{Lab}
Lab 7-1: Modeling a Controller
\begin{itemize}
\item Use the Verilog \textbf{case} statement to describe a controller
\end{itemize}
\begin{figure}
    \includegraphics[width=0.45\textwidth]{img/06_lab.png}
\end{figure}
\end{frame}

%%%%%%%%%%%%%%%%%%%%%%%%%%%%%%%%%%%%%%%%%%%%%%%%%%%%%%%%%%%%
\begin{frame}
\frametitle{Test Your Understanding - 1}
We are coding a conditional statement. We remember that if the expression value is not known then:
\begin{itemize}
\item[$\square$] first branch executes
\item[$\square$] both branches executes
\item[$\square$] second branch executes
\item[$\square$] neither branch executes
\end{itemize}
\end{frame}
\note{
We are coding a conditional statement. We remember that if the expression value is not known then:
\begin{itemize}
\item[$\square$] first branch executes
\item[$\square$] both branches executes
\item[$\boxtimes$] second branch executes
\item[$\square$] neither branch executes
\end{itemize}
}

%%%%%%%%%%%%%%%%%%%%%%%%%%%%%%%%%%%%%%%%%%%%%%%%%%%%%%%%%%%%
\begin{frame}
\frametitle{Test Your Understanding - 2}
We want to definitively test for a high-impedance output from the DUT. Which one or more statements can do that?
\begin{itemize}
\item[$\square$] casex
\item[$\square$] case
\item[$\square$] if
\item[$\square$] casez
\end{itemize}
\end{frame}
\note{
We want to definitively test for a high-impedance output from the DUT. Which one or more statements can do that?
\begin{itemize}
\item[$\square$] casex
\item[$\boxtimes$] case
\item[$\square$] if
\item[$\square$] casez
\end{itemize}
}

%%%%%%%%%%%%%%%%%%%%%%%%%%%%%%%%%%%%%%%%%%%%%%%%%%%%%%%%%%%%
\begin{frame}
\frametitle{Test Your Understanding - 3}
We can use the case, if and for statements in continuous assignment. Is it?
\begin{itemize}
\item[$\square$] True
\item[$\square$] False
\end{itemize}
\end{frame}
\note{
We can use the case, if and for statements in continuous assignment. Is it?
\begin{itemize}
\item[$\square$] True
\item[$\boxtimes$] False
\end{itemize}
}



\end{document}
