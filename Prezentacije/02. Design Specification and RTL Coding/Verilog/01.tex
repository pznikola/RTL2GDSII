\documentclass[t]{beamer}

% set fonts
\usefonttheme{professionalfonts} % using non standard fonts for beamer
\usepackage{txfonts,mathptmx}

% set indend spacing for first and second level indentation
\setlength{\leftmargini}{0.5cm}
\setlength{\leftmarginii}{0.5cm}

% Set circles for bullets 

\setbeamertemplate{itemize items}[circle]

% increase space between text and frame name
\addtobeamertemplate{frametitle}{}{\vspace{1em}}

%Information to be included in the title page:
\title{Verilog}
\author{Nikola Petrovic}
\institute{University of Belgrade, School of Electrical Engineering}
\date{2022}



\begin{document}

\frame{\titlepage}

%%%%%%%%%%%%%%%%%%%%%%%%%%%%%%%%%%%%%%%%%%%%%%%%%%%%%%%%%%%%
\begin{frame}
\frametitle{Course Prerequisite}

Before taking this course, you need to have:
\begin{itemize}
\item Basic computer literacy – you must know how to use a shell and editor of your choice and navigate the file system.
\item A basic understanding of digital hardware design and verification.
\item Knowledge of a procedural programming language will facilitate your learning experience.
\end{itemize}

\end{frame}

%%%%%%%%%%%%%%%%%%%%%%%%%%%%%%%%%%%%%%%%%%%%%%%%%%%%%%%%%%%%
\begin{frame}
\frametitle{Course Objective}

In this course, you: 
\begin{itemize}

\item Use fundamental Verilog language constructs required for design, verification and logic synthesis.
\item Analyze and ensure that Verilog designs meet the requirements for synthesis.
\item Compare mismatches that can happen between pre-synthesis and post-synthesis simulation and the process of synthesis.
\item Debug digital designs by developing Verilog test environments of significant capability and complexity.
\item Apply system tasks and functions to verify digital designs using the Incisive® Enterprise Simulator.
\end{itemize}

\end{frame}

%%%%%%%%%%%%%%%%%%%%%%%%%%%%%%%%%%%%%%%%%%%%%%%%%%%%%%%%%%%%
\begin{frame}
\frametitle{Course Agenda}
Day 1:
\begin{itemize}
\item About this course
\item Overview of cadence software
\item Descibing Verilog Applications
\item Verilog introduction
\item Choosing between Verilog Data Types
\item Using Verilog Operators
\end{itemize}

\end{frame}

%%%%%%%%%%%%%%%%%%%%%%%%%%%%%%%%%%%%%%%%%%%%%%%%%%%%%%%%%%%%
\begin{frame}
\frametitle{Course Agenda}
Day 2:
\begin{itemize}
\item Making Procedural Statements
\item Using Blocking and Non-Blocking Assignments
\item Using Continuous and Procedural Assignments
\item Understanding Simulation Cycle
Using Functions and Tasks
\item Directing the Compiler
\end{itemize}


\end{frame}

%%%%%%%%%%%%%%%%%%%%%%%%%%%%%%%%%%%%%%%%%%%%%%%%%%%%%%%%%%%%
\begin{frame}
\frametitle{Course Agenda}
Day 3:
\begin{itemize}
\item Introducing the Process of Synthesis
\item Coding RTL for Synthesis
\item Designing Finite State Machines
\item Avoiding Simulation Mismatches
\item Managing the Logic Synthesis process
\item Coding and Synthesizing an Example Verilog Design
\end{itemize}


\end{frame}

%%%%%%%%%%%%%%%%%%%%%%%%%%%%%%%%%%%%%%%%%%%%%%%%%%%%%%%%%%%%
\begin{frame}
\frametitle{Course Agenda}
Day 4:
\begin{itemize}
\item Using Verification Constructs
\item Coding Design Behavioural Algorithmically
\item Generating Test Stimulus
\item Developing a Testbench
\item Example Verilog Testbench
\item Course Conclusion
\item Next Steps
\end{itemize}


\end{frame}

\end{document}
